%!TEX root = 0卒業論文.tex
\newpage

\section{\rm 結言}
近年の情報化社会への急激な移り変わりや,IT技術者不足の背景などを受け,日本でも2020年より小学校でのプログラミング教育が必修化される.
英語教育では,2011年より小学校5学年以上で必修化されている.英語教育で躓かないようにするために,その前の学年向けの教育需要が生まれている.同じようにプログラミング教育においても小学校教育の前の段階である未就学児童向けの教育需要が生まれるのではないかと考えられる.

低年齢向けのプログラミング教材としては,ScrachやViscuitのような優秀なビジュアルプログラミング教材がすでに存在する.
しかし,これらの教材は,白紙のキャンパスと絵の具のようなものであり,自由度がとても高い.そのため学習の方向性が教材のみでは定めることが難しく,本人の理解力や想像力,やる気に学習体験が大きく左右されやすい.

また未就学児童は,教材を1人で操作し学習することが難しい.
そこで本研究では,小学校プログラミング教育への導入,プログラミング的思考力の向上,プログラミングの面白さの3点を満たす未就学児童でも学習することができる教材を制作を行った.

「桃太郎」をベースとした物語に沿った問題を出すことで,教材の操作を限定する.学習の方向性を教材側が定めることで未就学児でも操作可能な難易度にすることができる.既存の物語だけでなく,問題の解答結果によって物語が分岐する.解答に対するフィードバックを命令通りにキャラクターを動かすだけでなく,動作に結果によって,物語も変化させることでより体感的にプログラミングを感じることが出来る.
