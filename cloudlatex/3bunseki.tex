%!TEX root = 0卒業論文.tex

\newpage

\section{\rm 既存の教材の分析}

本章では,未就学児童向けの教材を制作するうえでどのような形式が良いか参考にするため,既存の教材と教材に組み込む要素を満たしているか判別をした.

\subsection{\rm 分析に用いた要素}
本研究で制作する教材に組み込む要素として上げた以下の7項目で判別を行った.\\\\
小学校プログラミング教育への導入\\
\UTF{2460}知識\\
問題の解決には必要な手順があることに気づくこと\\\\
\UTF{2461}思考力\\
発展の段階に即して,「プログラミング的思考」を育成すること\\\\
プログラミングの面白さ\\
\UTF{2462}ものを作ること\\\\
\UTF{2463}他人の役に立つこと\\\\
\UTF{2464}複雑な仕組みを見ること\\\\
\UTF{2465}新しいものを知ること\\\\
\UTF{2466}扱いやすい道具を使うこと\\


\subsection{分析対象の教材}
判別対象として以下の教材を使用した.\\\\
\begin{table}[htb]
\begin{center}
    \caption{判別対象の教材}
  \begin{tabularx}{\linewidth}{|X|X|} \hline
    教材名 & 概要 \\ \hline
    Scratch&ビジュアルプログラミングツール.命令ブロックを組み合わせてプログラミングをする \\ \hline
    Viscuit&ビジュアルプログラミング言語.変更前の絵,変更後の絵を組み合わせパラパラ漫画のように絵を動かしてプログラムをする\\ \hline
    プログル\cite{proguru} & 一般社団法人みんなのコードが提供するドリル型のプログラミング学習教材.Scrach に酷似している\\ \hline
    教育版レゴ\cite{lego}&レゴブロックの教育版である.ブロックをScrach のようなプログラミングツールで操る事ができる.一般的な高級言語でのプログラミング教育にも対応している\\ \hline
    CodeMonkey \cite{monkey}&ドリル型ゲームプログラミング教材. 課題をプログラムで解決する.独自言語をキーボードでコーディングする必要がある\\ \hline
    ルビィのぼうけん\cite{ruby} &プログラミング思考力,コンピュータの構造などを説明した絵本\\ \hline
    \end{tabularx}
  \label{tab:bamen1}
  \end{center}
\end{table}

\subsection{分析結果}
\begin{table}[htb]
\begin{center}
    \caption{教材に組み込む要素を満たしているか}
  \begin{tabular}{|c|c|c|c|c|c|c|c|} \hline
    教材& \UTF{2460} & \UTF{2461} & \UTF{2462}& \UTF{2463} & \UTF{2464}& \UTF{2465} & \UTF{2466}\\ \hline
    Scratch&△& ○ & ○& △& ○ & ○& ○\\ \hline
    Viscuit & △& ○ &○& △ & ○ & ○&○ \\ \hline
    プログル& ○ & ○ &△& △& ○ & ○&△ \\ \hline
    教育版レゴ& △ & ○ &○& △ & ○ & ○&○ \\ \hline
    CodeMonkey& ○ & ○ &△ & △ & ○ & ○&△ \\ \hline
    ルビィのぼうけん& ○ & ○ & × & △ & △ & ○&× \\ \hline

  \end{tabular}
  \label{tab:bamen1}
  \end{center}
\end{table}
\begin{center}
○:良 △:可 ×:不可\\
\end{center}

\subsection{考察}
7項目全てを満たす教材は無かった.プログラミングの面白さ「\UTF{2463}他人の役に立つこと」が,どの教材も苦手であるが教材であるため仕方ないのだと考える.問題に沿ってプログラムをしていくドリル型教材と自由にプログラムができるツール型教材との間に自由にものづくりができるという点から「\UTF{2460}知識」と「\UTF{2462}ものをつくる」という項目で差がついた.「\UTF{2460}知識」を満たす教材にするにはドリル的な部分が必要だと考える.絵本はその他教材の双方向的なフィードバックが得やすいものとは違い,一方向的な教材であるため満たす項目が少なかった.これらのことより小学校教育への導入要素である\UTF{2460},\UTF{2461}を満たすにはツール型ではなくドリル型である必要があると考える.

